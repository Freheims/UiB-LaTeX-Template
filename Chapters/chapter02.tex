% !TEX spellcheck = en-US
%%=========================================
\documentclass[../Main/thesis.tex]{subfiles}
\begin{document}
\chapter[Equations, etc]{Equations, Figures, and Tables}
\label{ch:equations}
The content of Chapter 2 will vary with the topic of your thesis. 
This chapter only gives guidance to some technical aspects of \LaTeX.
	 
\begin{remark}
If you want a shorter chapter or section title to appear in the Table of Contents and in the headings of the chapter, you just include the short title in square brackets before the title of the chapter/section. 
Example: \begin{verbatim}\section[Short Title]{Long Title}\end{verbatim}
\end{remark}

%%=========================================
\section{Simple Equations}
\label{sec:simple_equations}
Mathematical symbols and equations can written in the text as $\lambda$, $F(t)$, or even $F(t)=\int_0^t \exp(-\lambda x)\,dx$, or as displayed equations
\begin{equation}
F(t)=\int_0^t \exp(-\lambda x)\,dx
\label{eq:some_equation}
\end{equation}


The displayed equations are automatically given equation numbers -- here (\ref{eq:some_equation}) since this is the first equation in Chapter 2. 
Note that you can refer to the equation by referring to the ``label'' you specified as part of the equation environment.

You can also include equations without numbers:
\begin{equation*}
F(t)=\sum_{i=1}^n \binom{n}{i}\sin(i\cdot t)
\end{equation*}

%%=========================================
\subsection*{More Advanced Formulas}
Long formulas that cannot fit into a single line can be written by using the environment \texttt{align} as
\begin{align}
F(t)&= \sum_{i=1}^n \sin(t^{n-1}) - \sum_{i=1}^n \binom{n}{i}\sin(i\cdot t) \\
      & + \int_0^\infty n^{-x} e^{-\lambda x^t}\,dt
\end{align}

In some cases, you need to write ordinary letters inside the equations. 
You should then use the commands 
\begin{verbatim}
\textrm  and/or \mathrm
\end{verbatim}
The first command returns the normal text font and will be scaled automatically, while the second command will be scaled according to the use.
\begin{equation*}
\textrm{MTTF}= \int_0^\infty R_\mathrm{avg}(t)\,dt
\end{equation*}



Please consult the \LaTeX\ documentation for further details about mathematics in \LaTeX.
%%=========================================
\section*{Definitions}
\label{sec:definitions}
If you want to include a definition of a term/concept in the text, I have made the following macro (see in \texttt{uibstyle.sty}):
\begin{defin}
\textbf{Reliability}: The ability of an item to perform a required function under stated environmental and operational conditions and for a stated period of time.\newline
\end{defin}
When text is following directly after the definition, it may sometimes be necessary to end the definition text by the command
\begin{verbatim}
\newline
\end{verbatim}
I have not included this in the definition of the \texttt{defin} environment to avoid too much space when there is not a text-block following the definition.
%%=========================================
\section{Including Figures}
\label{sec:including_figures}
If you use pdf\LaTeX\ (as recommended), all the figures must be in pdf, png, or jpg format. 
We recommend you to use the pdf format.  
Please place the figure files in the directory \textbf{fig}. 
Figures are included by the command shown for Figure~\ref{fig:uib_logo_rotated}. 
Please notice the ``path'' to the figure file written by a \emph{forward} slash (/). 
You should not include the format of the figure file (pdg, png, or jpg) -- just write the ``name'' of the figure. 
\begin{figure}
\centering
\includegraphics[scale=0.6,angle=15]{../Figures/uib}
\caption{This is the logo of UiB (rotated 15 degrees).}
\label{fig:uib_logo_rotated}
\end{figure}

Each figure should include a unique \emph{label} as shown in the command for Figure~\ref{fig:uib_logo_rotated}. 
You can then refer to the figure by the \emph{ref} command.
Notice that you can scale the size of the figure by the option \texttt{scale=k}. 
You may also define a specific width or height of the figure by replacing the \texttt{scale} options by \texttt{width=k} or \texttt{height=k}. 
The factor \texttt{k} can here be specified in mm, cm, pc, and many other length measures. 
You may also give \texttt{k} as a fraction of the width of the text or of the height of the text, for example, \texttt{width=0.45$\backslash$textwidth}. 
If you later change the margins of the text, the figure width will change accordingly. 
As illustrated in Figure~\ref{fig:uib_logo_rotated}, you may also rotate the figure -- and also do many other things (please check the documentation of the package \texttt{graphicx} -- it is available on your computer, or you may find it on the Internet).

In \LaTeX\ all figures are floating objects and will normally be placed at the top of a page. 
This is the standard option in all scientific reports. 
If you insist on placing the figure exactly where you declare the figure, you may include the command \texttt{[h]} (here) immediately after $\backslash$\texttt{begin\{figure\}}. 
If you will force the figure to be located either at the top or bottom of the page, you may alternatively use  \texttt{[t]} or \texttt{[b]}. 
For more options, check the documentation.

Large figures may be included as a \emph{sidewaysfigure} as shown in Figure~\ref{fig:uib_logo}:\footnote{You can use a similar command for large tables.}
\begin{sidewaysfigure}
\centering
\includegraphics[scale=1.8]{../Figures/uib}
\caption{This is the logo of UiB.}
\label{fig:uib_logo}
\end{sidewaysfigure}

%%=========================================
\section{Including Tables}
\label{sec:including_tables}
\LaTeX\ has a lot of different options to include tables. 
Only one of them is illustrated here.

\begin{table}
	\centering\small
	\caption{The degree of newness of technology.}
	\label{tab:newness}
		\begin{tabular*}{\textwidth}{@{\extracolsep{\fill}}lccc}
			\toprule
			  &\multicolumn{3}{c}{Level of technology maturity}\\
  \cmidrule{2-4}
			Experience with the		   &  & Limited field history or not & New or \\
              operating  condition  & Proven &  used by company/user & unproven \\
        
			\midrule
			  Previous experience & 1 & 2 & 3 \\
		          No experience by company/user & 2 & 3 & 4 \\
		          No industry experience & 3 & 4 & 4 \\
			\bottomrule
		\end{tabular*}
\end{table}

\begin{remark}
Notice that figure captions (Figure text) shall be located \emph{below} the figure -- and that the caption of tables shall be \emph{above} the table. 
This is done by placing the $\backslash$\texttt{caption} command beneath the command $\backslash$\texttt{includegraphics} for figures, and above the command $\backslash$\texttt{begin\{tabular*\}} for tables.
\end{remark}
%%=========================================
\section{Copying Figures and Tables}
\label{sec:copying_figures_and_tables}
In some cases, it may be relevant to include figures and tables from from other publications in your report. 
This can be a direct copy or that you retype the table or redraw the figure. 
In both cases, you should include a reference to the source in the figure or table caption. 
The caption might then be written as: \textsl{Figure/Table xx: The caption text is coming here \citep{rausand04}.}

In other cases, you get the idea from a figure or table in a publication, but modify the figure/table to fit your purpose. 
If the change is significant, your caption should have the following format: \textsl{Figure/Table xx: The caption text is coming here \citep[adapted from][]{rausand04}.}

%%=========================================
\section{References to Figures and Tables}
\label{sec:references_to_figures_and_tables}
Remember that all figures and tables shall be referred to and explained/discussed in the text. 
If a figure/table is not referred to in the text, it shall be deleted from the report.
%%=========================================
\section{A Word About Font-encoding}
\label{sec:a_word_about_font-encoding}
When you press a button (or a combination of buttons) on your keyboard, this is represented in your computer according to the \emph{font-encoding} that has been set up. 
A wide range of font-encodings are available and it may be difficult to choose the ``best'' one. 
In the template, I have set up a font-encoding called UTF-8 which is a modern and very comprehensive encoding and is expected to be the standard encoding in the future. 
Before you start using this template, you should open the Preferences ->Editor dialogue in TeXworks (or TeXShop if you use a Mac) and check that encoding UTF-8 has been specified. 

If you use only numbers and letters used in standard English text, it is not very important which encoding you are using, but if you write the Norwegian letters æ, ø, å and accented letters, such as é and ä, you may run into problems if you use different encodings. 
Please be careful if you cut and paste text from other word-processors or editors into your \LaTeX\ file!

\subsubsection*{Warning}
If you (accidentally) open your file in another editor and this editor is set up with another font-encoding, your non-standard letters will likely come out wrong. 
If you do this, and detect the error, be sure \emph{not} to save your file in this editor!!

This is not a specific \LaTeX\ problem. 
You will run into the same problem with all editors and word-processors -- and it is of special importance if you use computers with different platforms (Windows, OSX, Linux).

%%=========================================
\section{Plagiarism}
\label{sec:plagiarism}
Plagiarism is defined as ``use, without giving reasonable and appropriate credit to or acknowledging the author or source, of another person's original work, whether such work is made up of code, formulas, ideas, language, research, strategies, writing or other form'', and is a very serious issue in all academic work. You should adhere to the following rules:
\begin{itemize}
\item Give proper references to all the sources you are using as a basis for your work. The references should be give to the original work and not to newer sources that mention the original sources.
\item You may copy paragraphs up to 50 words when you include a proper reference. In doing so, you should place the copied text in inverted commas (i.e., ``Copied text follows \ldots''). Another option is to write the copied text as a quotation, for example:
\begin{quote}
Birnbaum's measure of reliability importance of component $i$ at time $t$ is equal to the probability that the system is in such a state at time $t$ that component $i$ is critical for the system.\newline \mbox{} \hfill \citet{rausand04}
\end{quote}
\end{itemize}

\section{Code example}
\label{sec:code_example}
If you want to show some code in your thesis Minted is your friend.
\begin{listing}[ht]
\caption{Some code example}
\label{lst:clean.py}
\begin{minted}{python}
import os

os.chdir('test/')
backups = os.listdir().sort()
nrOfBackups = len(backups)
if(nrOfBackups > 10):
    oldestBackup = backups[0]
    os.removedirs(oldestBackup)
print(backups)
print(nrOfBackups)
\end{minted}
\end{listing}

\blankpage
\end{document}

